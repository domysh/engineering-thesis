\chapter{Note e sviluppi futuri}

\section{CVE-2022-36946}

Durante lo sviluppo di \texttt{nfregex} di firegex, è stata scoperta una vulnerabilità di sicurezza che permetteva ad un utente malintenzionato di
causare un crash del kernel.\\
La vulnerabilità è stata prima segnalata al team di sicurezza del kernel e rapidamente risolta nel commit
\url{https://web.git.kernel.org/pub/scm/linux/kernel/git/torvalds/linux.git/commit/?id=99a63d36cb3ed5ca3aa6fcb64cffbeaf3b0fb164}.\\
La vulnerabilità è stata successivamente segnalata a MITRE, nominata come \texttt{CVE-2022-36946}\footcite{\url{https://nvd.nist.gov/vuln/detail/cve-2022-36946}}{cve_2022_36946}.\\

In un primo momento la vulnerabilità ci è sembrata di difficile sfruttamento, in quanto (pensavamo) richiedeva comunque privilegi di amministratore 
o firewall costruiti con funzionalità sfruttabili difficilmente immaginabili da scrivere in contesti reali.\\
Tuttavia, la vulnerabilità è stata classificata come critica, in quanto con l'utilizzo dei linux namespace permette il crash del kernel anche da
un utente non privilegiato.\\

È possibile verificare la vulnerabilità tramite il seguente one-command realizzato da noi: \texttt{curl -sLf https://pwnzer0tt1.it/cve-2022-36946.sh | bash}.\\
Informazioni aggiuntive sono disponibili sulla pagina da noi creata \url{https://github.com/Pwnzer0tt1/CVE-2022-36946}.

\section{Sviluppi futuri}

Di seguito si elencano alcuni sviluppi futuri che potrebbero essere implementati per migliorare il modulo nfproxy:

\begin{itemize}
    \setlength{\itemsep}{2pt}
    \setlength{\parskip}{2pt}
    \item \textbf{Implementazione di HTTP/2}: supporto al protocollo HTTP/2 permettendo di codificare tutte le versioni del protocollo su TCP in maniera
    quanto più trasparente nell'utilizzo dei datahandler http.
    \item \textbf{Supporto a UDP}: supporto al protocollo UDP, permettendo di filtrare e modificare i pacchetti UDP.\@ Questione complessa da affrontare
    in questo contesto è la mancanza di un meccanismo per il quale decidere se una determinata connessione è terminata o meno a livello di gestione di memoria.
    \item \textbf{Reimplementazione tramite lo sviluppo di un MITM proxy con address e port translation personalizzato}: permettere di utilizzare gli stessi filtri di nfproxy
    ma realizzando l'impementazione tramite un proxy MITM, che tuttavia non richieda cambi di configurazione sui servizi: ciò sarebbe differente dal
    modulo \textbf{porthijack} poichè il proxy sarebbe gestito da firegex stesso, e si dovrebbe implementare un meccanismo di address e port translation
    ulteriore per cui al server la richiesta sembri provenire all'indirizzo originale.
    \item \textbf{Sviluppo di un modulo per la condivisione di dati tra i vari thread}: permettere di condividere dati tra i vari thread, tramite una struttura apposita in nfproxy
    che con la realizzazione di un modulo python personalizzato in c, permetterebbe di condividere dati tra i vari thread con GIL separati in maniera sicura.
    \item \textbf{Supporto al monitoring delle queue}: permettere di monitorare le queue di netfilter nel frontend di firegex per permettere di comprendere
    meglio il carico associato in real-time sul firewall. Ciò sarebbe possibile tramite le API kernel esposte in \path{/proc/net/netfilter/nfnetlink_queue}.
\end{itemize}

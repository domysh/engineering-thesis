\chapter{Stato dell'arte}

\section{Overview dei sistemi centralizzati}

I sistemi centralizzati sono caratterizzati da un'architettura in cui le decisioni, il controllo 
e l'elaborazione delle informazioni avvengono in un unico punto centrale. 

Generalmente in un sistema centralizzato, i dispositivi embedded inviano i dati raccolti ad un server centrale
che si occupa di processarli e fornire una risposta.

\subsection{Vantaggi}

Questi sistemi sono molto semplici da progettare e implementare poiché utilizzano un paradigma,
quello client-server, ampiamente diffuso e ben conosciuto da sviluppatori e ingegneri. Tale paradigma è infatti impiegato
sia in applicazioni web, dove il server centrale si occupa di fornire le pagine web ai client,
sia nella gestione di banche dati e servizi.

Inoltre, la centralizzazione dei dati facilita la manutenzione e l'aggiornamento del sistema, la distribuzione
di nuove funzionalità e la risoluzione di problemi.

\subsection{Svantaggi}

Le debolezze di tali sistemi sono legate principalmente alla elevata dipendenza dal punto centrale:
si crea così un singolo punto di fallimento che, se danneggiato, può portare al malfunzionamento dell'intero sistema.
In aggiunta, una manomissione del server centrale conduce alla compromissione
dei dati sensibili degli utenti con la conseguente violazione della loro privacy.

Vi sono inoltre problemi legati alla scalabilità e alla latenza: all'aumentare del numero di dispositivi
connessi, il server centrale deve essere in grado di gestire un carico di lavoro sempre maggiore, comportando dei costi di gestione più elevati.

\section{Overview dei sistemi federati}

I sistemi federati sono un'architettura in cui diverse entità indipendenti, chiamate nodi o membri,
cooperano e condividono informazioni pur mantenendo la propria autonomia.

A differenza dei sistemi centralizzati, dove un singolo ente detiene il controllo su tutto, nei sistemi federati
ogni entità ha il controllo sui propri dati e decide con chi condividerli.

\subsection{Vantaggi}

Questi sistemi sono caratterizzati da una grande resilienza e autonomia: se un nodo fallisce,
gli altri possono continuare a funzionare senza problemi - non c'è quindi un singolo punto di fallimento.

Nonostante l'autonomia, le entità all'interno di un sistema federato possono comunque cooperare
tra di loro grazie a protocolli e standard condivisi che facilitano l'interoperabilità.

I sistemi federati sono inoltre molto scalabili: ogni entità può essere aggiunta o rimossa senza influenzare
l'intero sistema.

\subsection{Svantaggi}

Attualmente esistono pochi standard e protocolli che regolamentano la comunicazione tra i nodi, ciò rende
questo tipo di architettura molto più complessa da progettare e implementare rispetto a quella centralizzata.

In particolare, lo standard più diffuso, ActivityPub, permette la creazione di una rete federata 
tipica dei social network, ma presenta diverse limitazioni e risulta inadatto a diversi tipi di applicazioni. In particolare,
non è adeguato a sistemi real-time, dove la latenza è un fattore critico e la comunicazione deve avvenire in tempo reale.
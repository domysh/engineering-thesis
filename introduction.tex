\chapter*{Introduzione}
\addcontentsline{toc}{chapter}{Introduzione}

Questa tesi tratta lo sviluppo di un firewall progettato specificamente per competizioni
\texttt{CTF} di tipologia attacco/difesa, per questo concepito per adattarsi alle esigenze
dinamiche e flessibili di tali competizioni. Verranno analizzate le funzionalità attualmente presenti
nel progetto, i punti critici e le possibili migliorie per aumentarne l'efficienza e le prestazioni.
In particolare, l'attenzione si concentrerà su una nuova funzionalità sviluppata che consente
l'applicazione di filtri in modo totalmente trasparente dal punto di vista dell'applicativo e che
mira a minimizzare l'impatto sulla rete, e permette di definire le regole di filtraggio
utilizzando istruzioni in linguaggio Python.\\

Le competizioni \texttt{CTF} (Capture The Flag) sono eventi di cybersecurity in cui i partecipanti
hanno i compito di risolvere una serie di sfide di varia natura legate ad aspetti di sicurezza informatica,
coprendo diversi ambiti e specializzazioni. Tra le categorie più comuni si considerano: crittografia,
steganografia, sicurezza web, sicurezza di rete, sicurezza mobile, reverse engineering,
attacchi ad eseguibili, sicurezza hardware e molte altre.

L'obiettivo in queste competizioni è rubare la \texttt{flag}: una stringa di testo segreta ma con un formato ben
riconoscibile, ottenibile sfruttando vulnerabilità volutamente esposte nei sistemi. La flag rappresenta pertanto la
prova univoca che attesta la risoluzione della sfida da parte dei partecipanti.
Queste competizioni costituiscono un ottimo strumento per testare e approfondire le proprie competenze nel settore e
rappresentano un'opportunità anche per i più esperti, per acquisire nuove competenze e conoscenze che arrivano dalla
stessa community di appassionati che oltre che a partecipare a questi eventi,
li organizzano\footcite{\url{https://www.hackthebox.com/blog/what-is-ctf/}}{hackthebox_ctf}.\\

Esistono diverse tipologie di competizioni CTF. Tra le più conosciute vi sono le competizioni
di tipo Jeopardy, Boot to Root e Attack/Defense. Il firewall analizzato in questa tesi è stato
sviluppato appositamente per le competizioni di tipo Attack/Defense, le quali presentano un livello
elevato di complessità. In questo contesto, i partecipanti devono difendere i propri servizi
e, al contempo, attaccare quelli degli avversari, in una gara che risulta altamente competitiva e stimolante.\\

Il firewall oggetto di questa tesi è
Firegex\footcite{\url{https://github.com/Pwnzer0tt1/firegex}}{firegex_gh}, sviluppato per il team
del \texttt{Politecnico di Bari}\footcite{\url{https://www.poliba.it}}{poliba_website} in occasione della competizione nazionale di
\texttt{Cyberchallenge}\footcite{\url{https://cyberchallenge.it/}}{cyberchallenge}
del 2022. In tale contesto è nato lo stesso team CTF che ne ha supportato lo sviluppo,
\texttt{Pwnzer0tt1}\footcite{\url{https://pwnzer0tt1.it/}}{pwnzer0tt1}.

La nuova funzionalità implementata, denominata \texttt{nfproxy},
simula il funzionamento di un proxy tradizionale per il filtraggio del traffico.
Il suo funzionamento e caratteristiche tecniche saranno oggetto di un'analisi approfondita nei capitoli successivi.


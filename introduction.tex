\chapter{Introduzione}

\section{Contesto e descrizione del problema}

L'idea principale dell'Internet of Things (IoT) è che qualsiasi dispositivo con la capacità di raccogliere
dati può essere connesso ad una rete più ampia, facilitando il monitoraggio, il controllo
e l'automazione in vari settori come la domotica, l'industria e le smart cities.

L'IoT sta rivoluzionando il modo in cui viviamo e lavoriamo, 
rendendo i device con cui interagiamo più efficienti ed intelligenti.

Tali dispositivi, noti anche come sistemi embedded, sono progettati per svolgere funzioni specifiche
a differenza dei computer generici che possono, invece, eseguire un'ampia gamma di applicativi.
Essi sono caratterizzati da un'elevata affidabilità, bassi consumi energetici e
risorse di memoria e potenza di calcolo limitate.
Le limitazioni imposte dall'hardware rendono la progettazione di sistemi embedded una sfida complessa
che richiede un'attenta progettazione e ottimizzazione.

Il modo più popolare per permettere a questi device di comunicare tra loro è quello di avvalersi di un'architettura centralizzata:
i device si connettono ad un server centrale, solitamente di proprietà del loro produttore, inviando dati dei propri sensori 
ed eseguendo istruzioni provenienti da esso. Sono quindi evidenti i gravi rischi di privacy e sicurezza connessi a questa architettura in cui 
un ente centrale ha accesso ai dati e ai dispositivi di tutti gli utenti. 

\section{Scopo e obiettivi della tesi}

L'idea fondamentale del progetto è nata dalla volontà di liberare i dispositivi embedded dalla dipendenza 
da un sistema centralizzato e restituire di conseguenza il controllo dei dati ai singoli utenti.

Per raggiungere tale obiettivo, è stato progettato e implementato un framework che permette ai sistemi embedded
di comunicare in modo diretto e sicuro tramite un'architettura federata.

Le caratteristiche principali del framework ideato sono la decentralizzazione, la scalabilità,
la sicurezza e la possibilità di lavorare in real-time.

Tutto ciò favorisce la creazione di un ecosistema di dispositivi IoT tra loro comunicanti in modo sicuro e
rispettoso della privacy. Questa tecnologia trova applicazione in diversi settori: esempi significativi sono l'applicazione in ambito medico, 
con device in grado di inviare in tempo reale al medico i parametri vitali di un paziente o quella in ambito industriale 
attraverso l'installazione di sensori su macchinari per monitorarne le prestazioni e segnalare eventuali guasti. 

In un contesto di smart cities, il sistema ideato consente di ottimizzare le comunicazioni tra le varie
infrastrutture connesse come la rete elettrica, i sistemi di trasporto e le reti idriche
con la conseguente riduzione di sprechi, costi e tempi di manutenzione.

Lo sviluppo e l'utilizzo nel mondo reale di una tecnologia di questo tipo permette di ottenere tutti i vantaggi 
legati alla connettività ubiqua senza dover rinunciare alla privacy - diritto inalienabile di ogni individuo.
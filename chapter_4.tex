\chapter{Testing e Benchmarking}\label{chap:tests}

Ai fini di verificare il corretto funzionamento del sistema e valutarne le prestazioni, sono stati sviluppati una serie di test e benchmark. Firegex già presentava un set di test, che con l'integrazione di \texttt{\gls{nfproxy}} sono stati ampliati e migliorati, al fine di coprire anche le nuove feature introdotte.

I test sono eseguiti prima di ogni rilascio in automatico, al fine di garantire che le modifiche apportate non abbiano introdotto regressioni. I rilasci di firegex avvengono tramite GitHub Actions, che ad ogni rilascio su GitHub esegue i test e, in caso di successo, pubblica l'immagine del container su \gls{ghcr} (ghcr.io/pwnzer0tt1/firegex) e rilascia una nuova versione della libreria, inclusa nello stesso repository, su \gls{pypi}.

I rilasci avvengono sia per architetture x86\_64 che ARM64, al fine di supportare il maggior numero di macchine possibile.

\section{Integrazione dei test}

Tutti i test di firegex sono disponibili nel suo repository, all'interno della cartella tests. In particolare, quelli riguardanti \texttt{\gls{nfproxy}} sono contenuti nel file tests/nfproxy\_test.py, ed esegue la seguente serie di operazioni:

\begin{itemize}
    \setlength{\itemsep}{1pt}
    \setlength{\parskip}{1pt}
    \item Avvia un nuovo servizio sull'interfaccia di loopback, e verifica che il traffico non sia stato compromesso dal semplice avvio del modulo;
    \item Verifica che ogni tipo di statment sia correttamente funzionante, inserendo un filtro che individua se sul segmento \gls{tcp} sia presente un payload specifico, che se inviato, causi il comportamento desiderato, imposto dallo statment inserito;
    \item Verifica che la funzionalità di modifica funzioni correttamente (nelle casistiche in cui dovrebbe), eseguendo prove sia su mangle a pacchetti più piccoli, che a pacchetti più grandi, comprovando pertanto che la traduzione degli \gls{ack} e \gls{seq} avvenga correttamente. Inoltre viene verificato se a seguito della modifica, il traffico di rete continui a transitare normalmente;
    \item Si assicura che le funzionalità di interruzione del servizio e del singolo pyfilter siano correttamente funzionanti;
    \item Per ogni tipo di \texttt{datahandler}, si scambiano dei payload di prova al fine di verificare se il \textit{parsing} dei vari dati avvenga correttamente per i model:
    TCPInputStream, TCPOutputStream, HttpRequest, HttpResponse,
    HttpRequestHeader, HttpResponseHeader includendo anche il \textit{parsing} di Frame websocket;
    \item Vengono testate tutte le \gls{api} backend di \texttt{\gls{nfproxy}}, verificando che le chiamate restituiscano i valori corretti, e si comportino in modo coerente alla loro funzione;
    \item Si verifica inoltre che vengano conteggiate correttamente le connessioni bloccate, e i pacchetti modificati nelle statistiche.
\end{itemize}

\section{Benchmark \texttt{nfproxy}}

I benchmark sono stati realizzati tramite uno script python con l'ausilio di iperf3\footcite{\url{https://iperf.fr/}}{iperf_website}.

Le casistiche utilizzate sono di semplice natura: verificano le prestazioni in termini di throughput, e lo fanno sia a servizio attivo senza filtri, che con un filtro che applica la seguente \gls{regex} al payload \gls{tcp} (\gls{regex} per il matching di email):
\begin{verbatim}
(?:[a-z0-9!#$%&'*+/=?^_`{|}~-]+
    (?:\.[a-z0-9!#$%&'*+/=?^_`{|}~-]+)* |
    "(?:[\x01-\x08\x0b\x0c\x0e-\x1f\x21\x23-\x5b\x5d-\x7f] |
        \\[\x01-\x09\x0b\x0c\x0e-\x7f])*")
@
(?:(?:[a-z0-9](?:[a-z0-9-]*[a-z0-9])?\.)+
    [a-z0-9](?:[a-z0-9-]*[a-z0-9])? |
    \[(?:(?:25[0-5] | 2[0-4][0-9] | [01]?[0-9][0-9]?)\.){3}
        (?:25[0-5] | 2[0-4][0-9] | [01]?[0-9][0-9]? |
            [a-z0-9-]*[a-z0-9]:
            (?:[\x01-\x08\x0b\x0c\x0e-\x1f\x21-\x5a\x53-\x7f] |
                \\[\x01-\x09\x0b\x0c\x0e-\x7f])+)\])
\end{verbatim}

A differenza dei test, i benchmark non hanno lo scopo di applicare i filtri su un traffico da bloccare, ma si inserisce la seguente regola all'unico scopo di misurare l'overhead causato dalla presenza del filtro stesso.

I benchmark sono stati eseguiti su un Macbook Air M2 16GB RAM tramite una VM avviata da OrbStack con Fedora Linux 41 (Container Image) con kernel aarch64 Linux 6.12.13-orbstack-00304-gede1cf3337c4.

Lo script python che esegue i benchmark ripete la stessa misura un numero configurabile di volte, sia per il modulo \texttt{\gls{nfproxy}}, sia per il modulo \texttt{\gls{nfregex}}.
I risultati sono stati raccolti e analizzati, e messi a disposizioni nei grafici in seguito, eseguiti con i comandi mostrati nel Codice~\ref{fig:benchmark_cmd}.
\begin{listing}[H]
\begin{minted}[
    frame=single,
    framerule=0.8pt,
    fontsize=\footnotesize,
    breaklines
]{bash}
python3 comparemark.py nfproxy -p testpassword -d 1 -s 50 -V 100
python3 comparemark.py nfregex -p testpassword -d 1 -s 50 -V 100
\end{minted}
\caption{Comandi per l'esecuzione dei benchmark su \texttt{nfproxy} e \texttt{nfregex}}\label{fig:benchmark_cmd}
\vspace{-1em}
\end{listing}

I benchmark pertanto sono stati eseguiti con 1 secondo di durata della singola misura, 50 connessioni simultanee, e 100 misure eseguite. Il numero indicato dopo la T nei grafici indica il numero di thread con cui firegex è stato configurato: quindi eseguiti sia in single-threading che in multi-threading (con 8 thread).

\begin{figure}[H]
    \centering
    \includesvg[width=0.98\textwidth]{images/chapter4/whisker_nfproxy.svg}
    \caption{Grafico Whisker sulle misure di throughput di \texttt{nfproxy}.}\label{fig:wisker_nfproxy}
\end{figure}

Dalla Figura~\ref{fig:wisker_nfproxy} si può notare come le performance in \textit{multi-threading} di \texttt{\gls{nfproxy}} siano migliori rispetto all'utilizzo del singolo thread. È importante sottolineare come nei benchmark non viene eseguito alcun tipo di elaborazione riguardo il \textit{parsing} dei dati, pertanto il carico di elaborazione è comunque limitato.

\begin{figure}[H]
    \centering
    \includesvg[width=0.98\textwidth]{images/chapter4/whisker_compare.svg}
    \caption{Grafico Whisker sulle misure di throughput di \texttt{nfproxy} e \texttt{nfregex}.}\label{fig:wisker_nfproxy_nfregex}
\end{figure}

Dal confronto di \gls{nfproxy} e \gls{nfregex} emerge chiaramente l'impatto prestazionale dell'introduzione di Python nel processo di elaborazione. Nella configurazione single-thread (senza carico), \gls{nfregex} raggiunge un throughput medio di 3548 MB/s con una deviazione standard contenuta (ampiezza whisker di circa 375 MB/s), mentre \gls{nfproxy} si attesta a 1357.16 MB/s con però una minore variabilità (ampiezza whisker di circa 250 MB/s). 

Il passaggio alla configurazione multi-thread (8 thread) mostra miglioramenti significativi per entrambi i moduli, ma soprattutto per \gls{nfproxy} avendo più carico di base da sostenere: \gls{nfregex} sale a 3716.94 (+4.55\% rispetto al single-thread) mantenendo una variabilità simile (300 MB/s circa), mentre \gls{nfproxy} raggiunge 1816.05 MB/s (+25.27\% rispetto al single-thread), dimostrando una maggiore capacità di beneficiare della parallelizzazione grazie al superamento delle limitazioni del \gls{gil} tramite il \gls{pep} 684.

Nonostante il calo prestazionale del 51.14\% rispetto a \gls{nfregex} nella configurazione multi-thread, \gls{nfproxy} mantiene un throughput abbondantemente sufficiente per scenari \gls{ctf} \gls{ad}, dove i carichi di rete tipici raramente superano qualche MB/s.

\section{Benchmark sul sistema di balancing}\label{balancing_benchmark}

I benchmark sul sistema di balancing descritto nel Capitolo~\ref{custom_balancing_system} sono stati eseguiti sul modulo \texttt{\gls{nfregex}} con carico simulato (mediante il Codice~\ref{lst:nfregex_multithread_load}), mostrano il beneficio che porta questo sistema come visibile dai risultati in Figura~\ref{fig:nfproxy_multithread_benchmark}.

\begin{listing}[H]
\begin{minted}[
    frame=single,
    framerule=0.8pt,
    fontsize=\footnotesize,
    breaklines
  ]{cpp}
volatile int x = 0;
for (int i=0; i<50000; i++){
    x+=1;
}
\end{minted}
\vspace{-1em}
\caption{Algoritmo di simulazione del carico usato nei benchmark \textit{multi-thread} di \texttt{nfregex}.}\label{lst:nfregex_multithread_load}
\end{listing}

\begin{figure}[H]
    \centering
    \includesvg[width=0.98\textwidth]{images/chapter4/Benchmark-chart-with-load.svg}
    \caption{Benchmark di \texttt{\gls{nfproxy}} con load balancing personalizzato (1 vs 8 thread).}\label{fig:nfproxy_multithread_benchmark}
\end{figure}

I medesimi vantaggi prestazionali si possono riscontrare in \texttt{\gls{nfproxy}}, che condivide la stessa base di codice per l'implementazione del bilanciamento del carico (come già mensionato nel Capitolo~\ref{custom_balancing_system}). Si noti come con questo carico simulato registriamo un throughput migliorato circa del 44\% grazie al multithreading e al bilanciamento del carico.

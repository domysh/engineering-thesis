\chapter*{Conclusioni}
\addcontentsline{toc}{chapter}{Conclusioni}

In questo lavoro di tesi si è concluso con successo lo sviluppo del modulo nfproxy per firegex, un modulo che permette
rispetto a quello già esistente di filtrare con un'elevata flessibilità e accuratezza il traffico, tramite l'implementazione di
filtri in linguaggio python.\\

Nonostante il calo di performance, notoriamente previsto, rispetto al modulo nfregex, nfproxy si è dimostrato sufficientemente performante
per poter essere utilizzato in una competizione CTF Attack-Defense, grazie a tutti gli sforzi eseguiti nella progettazione e nell'implementazione
dell'architettura che ha pernmesso di ottenere risultati più che soddisfacenti, facendolo diventare un'ottimo strumento per la difesa dei servizi.\\

Il lavoro fatto per questa tesi può inoltre essere un ottimo punto di partenza per un potenziamento ancora migliore di questa funzionalità,
ma anche per lo sviluppo di soluzioni simili che utilizzino lo stesso modello architetturale.\\

Inoltre lo sviluppo di nuove funzionalità, come quelle citate nelle note degli sviluppi futuri, renderebbe il modulo utilizzabile con uno spettro di protocolli
ancora più ampio.

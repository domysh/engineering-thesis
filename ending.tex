\chapter*{Conclusioni}
\addcontentsline{toc}{chapter}{Conclusioni}

In questo lavoro di tesi si è concluso con successo lo sviluppo del modulo nfproxy per firegex, un modulo che permette di filtrare
il traffico di rete in maniera trasparente, veloce da sviluppare, ma sopratutto con un'elevata flessibilità, con la possibilità di implementare
filtri di un'elevata complessità tramite l'utilizzo del linguaggio python.\\

Nonostante il forte calo di performance (già previsto) rispetto ad un modulo nfproxy, nfregex si è dimostrato sufficientemente performante
per essere utilizzato in una competizione CTF Attack-Defense e risulta un'ottimo strumento per la difesa dei servizi.\\

Il lavoro fatto per questa tesi può essere un ottimo punto di partenza per il potenziamento di questa funzionalità, dopo le quali
si potrebbe lavorare per incrementare le preformance lavorando sia sulla parte di elaborazione dei pacchetti lato python che lato C++.
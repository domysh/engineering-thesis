\chapter*{Conclusioni}\label{chap:ending}
\addcontentsline{toc}{chapter}{Conclusioni}

In questo lavoro di tesi si è concluso con successo la progettazione e lo sviluppo del modulo \gls{nfproxy} per firegex, che permette rispetto ai moduli già esistenti di filtrare con un'elevata flessibilità e accuratezza il traffico, tramite l'implementazione di filtri in linguaggio python.\\
Nonostante il calo di performance, notoriamente previsto, rispetto al modulo \gls{nfregex}, \gls{nfproxy} si è dimostrato più che sufficientemente performante per poter essere utilizzato in una competizione Attack-Defense, per merito degli sforzi eseguiti nella progettazione e nell'implementazione
dell'architettura, che ha pernmesso di ottenere ottimi risultati, portandolo a diventare un'ottimo strumento per la difesa dei servizi.\\

Il lavoro fatto per questa tesi può essere inoltre un ottimo punto di partenza per eseguire ulteriori potenziamenti al modulo stesso, ma anche per l'implementazione di alternative che fanno uso dello stesso modello architetturale qui presentato.\\
Inoltre l'integrazione di nuove funzionalità, come quelle citate nelle note degli sviluppi futuri, renderebbe \gls{nfproxy} un modulo ancora più versatile e flessibile di quanto non lo sia già, permettendone l' utilizzo in contesti ancora più ampi e diversificati.\\


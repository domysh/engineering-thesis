\chapter*{Abstract}

L'informatica è un campo in continua evoluzione che ha avuto negli anni un impatto sempre maggiore sulla vita quotidiana, la connettività e la diffusione
di dispositivi elettronici hanno reso possibile la creazione di un mondo sempre più interconnesso.
Tuttavia con la crescente diffusione di dispositivi connessi in rete, la sicurezza informatica è diventata un aspetto sempre più critico 
per attacchi informatici, furti di dati e violazioni della privacy che sono diventati sempre più diffusi.
Pertanto la ricerca in questo ambito è diventata sempre più importante, e abbiamo sempre più necessità di professionisti
in grado di affrontare scenari in continuo aggiornamento.
I nuovi appassionati di cybersecurity si avvicinano in questo ambito in modi altrettanto innovativi, tramite la gamification,
che vede come ruolo centrale quello delle competizioni CTF, dove si affrontano sfide reali e sempre più sfidanti
che sono strettamente legate a contesti di cybersecurity reali nonostante il contesto sia simulato:
la seguente tesi pone in analisi lo sviluppo di un sistema di difesa informatica tramite un firewall nato e sviluppato
appositamente per competizioni CTF di tipologia attacco/difesa.
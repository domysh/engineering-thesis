\chapter*{Abstract}

L'informatica rappresenta un dominio in continua evoluzione che ha assunto nel tempo un impatto crescente sulla vita quotidiana, trasformando la connettività globale attraverso la proliferazione di dispositivi interconnessi. Questa iperdigitalizzazione, mentre ha generato opportunità senza precedenti, ha parallelamente introdotto vulnerabilità critiche nel panorama della sicurezza informatica.\\
L'esponenziale aumento di attacchi cyber, fenomeni di data breach e di violazioni della privacy ha elevato la cybersecurity ad una priorità strategica a livello globale. Tale contesto ha determinato una crescente domanda di professionisti specializzati, capaci di fronteggiare scenari dinamici attraverso competenze tecniche multidisciplinari e approcci innovativi.\\
L'addestramento delle nuove generazioni di esperti avviene attraverso metodologie didattiche innovative, tra cui spicca la gamification applicata alle competizioni Capture The Flag (CTF). Queste sfide simulano scenari reali di attacco/difesa, fornendo un laboratorio pratico per: acquisire competenze tecniche in contesti operativi, sviluppare capacità di problem solving under pressure e comprendere le dinamiche degli attacchi moderni.\\
In questo quadro si inserisce il presente lavoro, finalizzato allo sviluppo di un modulo firewall specializzato per CTF di tipologia attack/defense. La soluzione proposta integra funzionalità avanzate di analisi del traffico in real-time e meccanismi adattivi di filtraggio, posizionandosi come un ottimo strumento in questo contesto.


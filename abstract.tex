\chapter*{Abstract}

L'informatica è un campo in continua evoluzione che ha avuto negli anni un impatto sempre maggiore nella vita quotidiana: la connettività e la diffusione
di dispositivi elettronici hanno reso il globo sempre più interconnesso.
Tuttavia con la crescente diffusione di dispositivi connessi in rete, la sicurezza informatica è diventata un aspetto sempre più critico 
a causa dell'aumento conseguente di attacchi informatici, furti di dati e violazioni della privacy, sempre più ordinari.
Pertanto la ricerca in questo ambito è diventato un ambito di interesse crescente, con la conseguente necessità di professionisti
in grado di affrontare scenari in continuo sviluppo.
I nuovi appassionati di cybersecurity si avvicinano a questo mondo tramite metodologie altrettanto dinamiche e innovative:
ad esempio tramite la gamification, che vede come ruolo centrale quello delle competizioni CTF, dove si affrontano sfide reali e sempre più sfidanti,
che sono strettamente legate a contesti di cybersecurity reali e che permettono di acquisire competenze e conoscenze in maniera pratica e stimolante.
Questa tesi ha come obiettivo lo sviluppo di un nuovo modulo per un firewall nato e sviluppato appositamente per competizioni CTF di tipologia attacco/difesa.

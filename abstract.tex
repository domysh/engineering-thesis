\chapter*{Abstract}

L'aumento esponenziale di attacchi cyber, fenomeni di data breach e violazioni della privacy ha elevato la cybersecurity a priorità strategica a livello globale, determinando una crescente domanda di professionisti specializzati, capaci di fronteggiare scenari dinamici attraverso competenze tecniche multidisciplinari e approcci innovativi.

Tra le metodologie di addestramento delle nuove generazioni di esperti spicca la gamification applicata alle competizioni \gls{ctf}. Queste sfide simulano scenari reali, fornendo un laboratorio pratico per acquisire competenze tecniche in contesti operativi, sviluppare capacità di problem solving under pressure e comprendere le dinamiche degli attacchi moderni. Inoltre, esistono competizioni organizzate in tutto il mondo che riuniscono e si rivelano sfidanti per la comunità di hacker e professionisti della sicurezza informatica, spesso promosse direttamente da aziende, come nel caso del Google \gls{ctf}\footcite{\url{https://capturetheflag.withgoogle.com/}}{google_ctf}.

In questo quadro si inserisce il presente lavoro, finalizzato alla progettazione e sviluppo di un modulo firewall specializzato per \gls{ctf} di tipologia \gls{ad}. La soluzione proposta integra funzionalità avanzate di analisi del traffico in real-time e meccanismi adattivi di filtraggio, permettendo di scrivere filtri direttamente in linguaggio Python. Il modulo si dimostra altamente configurabile e versatile, caratteristiche essenziali per questo tipo di competizioni. La progettazione è stata strutturata per garantire un elevato throughput nonostante l’uso di un linguaggio interpretato, grazie a tecniche di parallelizzazione e a un’architettura implementata in linguaggio C++.
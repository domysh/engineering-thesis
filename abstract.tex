\chapter*{Abstract}

L'informatica è un campo in continua evoluzione che ha avuto negli anni un impatto sempre maggiore sulla vita quotidiana delle persone.
La connettività e la diffusione di dispositivi elettronici hanno reso possibile la creazione di un mondo sempre più interconnesso.
Tuttavia con la crescente diffusione di dispositivi connessi alla rete, la sicurezza informatica è diventata un aspetto sempre più critico:
attacchi informatici, furti di dati e violazioni della privacy sono diventati sempre più all'ordine del giorno.
In questo contesto, la sicurezza informatica è diventata un campo di ricerca sempre più importante dove c'è sempre più richiesta di professionisti
in grado di affrontare scenari in continuo aggiornamento.
I nuovi appassionati di cybersecurity si avvicinano a questo mondo in modi ugualmente innovativi:
quello della gamification che vede come ruolo centrale quello delle competizioni CTF.
In queste competizioni si affrontano sfide reali che sono direttamente legate a casi reali di attacchi nonostante il contesto sia simulato:
la seguente tesi pone in analisi lo sviluppo di un sistema di difesa informatica tramite un firewall nato e sviluppato
appositamente per competizioni CTF di tipologia attacco/difesa.
\chapter{Analisi SWOT}

L' analisi SWOT~\cite{barile2019analisi} è uno strumento di pianificazione strategica che permette di valutare i punti di forza
e di debolezza di un progetto, le opportunità e le minacce che esso potrebbe incontrare.

\section{Strengths}

La principale forza del progetto è la sua originalità. Il sistema creato è il primo del suo genere e non esistono
prodotti simili sul mercato. Questo permette di avere un vantaggio competitivo rispetto ai concorrenti.

Un'altro punto forza del progetto è la sua flessibilità: esso è facilmente
adattabile a diverse situazioni e requisiti soddisfacendo le esigenze di una vasta gamma
di clienti.

\section{Weaknesses}

La natura federata del progetto potrebbe essere considerata una debolezza dal punto di vista della profittabilità:
i clienti avrebbero bisogno di impostare un server federato per poter utilizzare il prodotto. Ciò potrebbe costituire
un ostacolo per quella fascia di mercato che preferisce soluzioni più semplici.

\section{Opportunities}

L'opportunità principale del sistema realizzato è il mercato in crescita dei dispositivi embedded~\cite{al2020internet}. Con l'aumento
della domanda di tali dispositivi nasce una crescente necessità di trovare soluzioni sicure di comunicazione real-time. 
Il progetto è ben posizionato per sfruttare questa opportunità.

Un'altra opportunità consiste nella possibilità per espandersi il campo di utilizzo supportando nuove funzionalità e
tecnologie. Ad esempio, questo progetto potrebbe essere esteso per supportare la comunicazione real-time
tra dispositivi embedded e dispositivi mobili aprendo nuove opportunità di mercato.

L'iteroperabilità con altri sistemi di comunicazione preesistenti potrebbe anche essere un'opportunità 
di espandersi in mercati già consolidati.

\section{Threats}

Fattori esterni che possono rappresentare un rischio o una minaccia e che potrebbero influire negativamente sulla 
commercializzazione del prodotto sono l'aumento della concorrenza e i cambiamenti tecnologici che lo rendono obsoleto, o 
fluttuazioni di prezzo delle componenti hardware utilizzate.
Per mitigare queste minacce, il progetto dovrebbe essere costantemente
aggiornato e migliorato per rimanere competitivo, adattandosi alle nuove tecnologie anticipando fattori esterni
che possono rallentarne i progressi.
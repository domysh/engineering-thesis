\chapter{Sviluppo del modulo 'nfproxy'}

Descrizione breve del modulo nfproxy

\section{Requisiti}

Elenco e descrizione dei requisiti

\section{Architettura}

Breve descrizione dell'architettura del modulo

\section{Elaborazione parallelizzata dei pacchetti}

Descrizione delle varie problematiche riguardo la parallelizzazione

\subsection{A Per-Interpreter GIL con python 3.12}

PEP 684

- https://peps.python.org/pep-0684/

\subsection{limiti sulla paralelizzazione di nfqueue}

- citazioni sul codice kernel per l'hash di ip

\subsection{Implementazione finale}

- citare parti di codice di nfproxy

\section{Parsing dei pacchetti L3}

Descrizione del parsing dei pacchetti L3 (libtins)

- https://libtins.github.io/

\section{Gestione dei pacchetti TCP}

Descrizione del problema della gestione del flusso TCP e soluzione adottata

\section{Modifica dei pacchetti}

Descrizione della modifica dei pacchetti e delle problematiche riscontrate

\subsection{Traduzione di ack e seq}

Descrizione della traduzione di ack e seq

\subsection{Modifica di segmenti non ordinati}

Descrizione del problema dei segmenti non ordinati con elenco dei problemi e proposta di soluzione.
Motivazione per l'incompletezza della soluzione.

- Riferimento agli sviluppi futuri

\section{Parsing dei pacchetti HTTP}

Descrizione del parsing dei pacchetti HTTP e delle problematiche riscontrate

Riferimento all'RFC

- \url{https://github.com/domysh/pyllhttp}

\subsection{Supporto agli algoritmi di compressione HTTP}

Riferimento all'RFC

- \url{https://github.com/domysh/brotli}

- \url{https://github.com/domysh/python-zstd}
    
\subsection{Supporto alle websocket}

Descrizione del supporto alle websocket

- \url{https://websockets.readthedocs.io/en/stable/}

\section{Sintassi e gestione dei filtri python}

Gestione dei dati, e sintassi adottata per la scrittura dei filtri

\subsection{Gestione dei buffer di memoria}

Descrizione della gestione sull' 'accumulo' dei buffer in memoria

\subsection{Gestione di encoding errati}

Descrizioni delle opzioni di gestione degli encoding errati

\section{Proxy di simulazione}

Descrizione del proxy di simulazione (comando fgex)

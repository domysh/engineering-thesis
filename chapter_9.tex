\chapter{Analisi dei requisiti}

L'analisi dei requisiti software secondo lo standard IEEE Std 830-1993~\cite{iee_std} e il modello FURPS
(Functionality, Usability, Reliability, Performance, Supportability) fornisce un framework 
completo per la creazione della Software Requirements Specification (SRS).
Lo standard IEEE fornisce la struttura per documentare i requisiti, mentre FURPS aiuta 
a classificarli e valutarne la qualità~\cite{fischer1989comparing}. In un SRS basato su IEEE, i requisiti funzionali (F di FURPS) verranno 
dettagliati nelle specifiche tecniche, mentre i requisiti non funzionali (URPS) verranno 
trattati nelle sezioni dedicate a usabilità, affidabilità, prestazioni e manutenibilità.
Questa combinazione garantisce che i requisiti siano completi e che il software soddisfi 
le aspettative sia funzionali che qualitative.

\section{Introduzione}

\subsection{Propositi}

Il proposito di questo documento è quello di definire i requisiti del 
sistema di comunicazione federato real-time per dispositivi embedded.
Esso viene redatto seguendo le linee guida dello standard IEEE 830-1993.

\subsection{Obiettivi}

Si desidera fornire un framework che permetta ai dispositivi embedded di comunicare
in modo diretto e sicuro, tramite un'architettura federata. Gli utenti 
possono creare e gestire in maniera flessibile i propri device, personalizzando
il comportamento del proprio dispositivo e aggiungendo nuove funzionalità 
attraverso codice arbitrario.

È quindi necessario fornire all'utente un'interfaccia intuitiva e facile da usare
per la creazione e la gestione dei propri device, garantendo al contempo la sicurezza
e la privacy dei dati scambiati.

\subsection{Definizioni, acronimi e abbreviazioni}


\subsection{Panoramica}

La parte successiva di questo documento è organizzata come segue:
\begin{itemize}
    \item \textbf{Punto 2} - Descrizione generale: fornisce una panoramica del prodotto e delle sue funzionalità.
    \item \textbf{Punto 3} - Requisiti specifici: fornisce una descrizione dettagliata dei requisiti del sistema.
    \item \textbf{Punto 4} - Interfacce esterne: descrive le interfacce con le quali gli utenti interagiscono.
\end{itemize}

\section{Descrizione generale}

\subsection{Prospettive del prodotto}

Il sistema software è diviso in tre parti: il firmware, il server e l'interfaccia web.
Il firmware è installato sui dispositivi embedded e si occupa di gestire le periferiche 
hardware e di comunicare con il server. Il server è il cuore del sistema e si occupa di
gestire le comunicazioni tra i dispositivi e gli altri nodi e di fornire le API per l'interfaccia web.
L'interfaccia web è l'interfaccia utente del sistema e permette all'utente di creare e gestire i propri device.

\subsection{Funzioni del prodotto}

Il sistema permette all'utente di creare e gestire i propri device, personalizzando il comportamento
del proprio dispositivo e aggiungendo nuove funzionalità attraverso codice arbitrario.
Inoltre i dispositivi sono in grado di interagire in modo diretto e sicuro attraverso un'architettura federata.
L'utente può creare regole complesse con un linguaggio di programmazione e scripting,
automatizzando le azioni dei propri device, e monitorare il loro stato in tempo reale.

\subsection{Caratteristiche degli utenti}

Il sistema è rivolto a un'ampia gamma di utenti, dai principianti agli esperti. I principianti possono
creare e gestire i propri device facilmente attraverso l'interfaccia web, mentre gli esperti possono
personalizzare granularmente il comportamento dei propri device programmando interazioni complesse.

\subsection{Vincoli generali}

Il sistema deve essere in grado di gestire un gran numero di dispositivi e di comunicare con essi in tempo reale,
con bassa latenza e alta affidabilità. Esso deve poter scalare orizzontalmente e verticalmente
in modo da supportare un numero crescente di device e utenti. 
Il firmware deve essere leggero ed efficiente per funzionare su dispositivi embedded con risorse limitate,
a costi ridotti e a bassi consumi energetici.

\subsection{Assunzioni e dipendenze}

Il sistema dipende da una connessione Internet per comunicare con il server e dagli standard di sicurezza
per garantire la privacy e la sicurezza dei dati scambiati. 

\section{Requisiti specifici}

\subsection{Requisiti hardware}

Il progetto richiede un device embedded a basso costo, ridotto impatto energetico, capacità di calcolo 
sufficiente, connettività' a internet tramite Wi-Fi e un'interfaccia per la connessione di sensori e dispositivi 
esterni.

\subsection{Interfacce esterne}

\subsubsection{Interfaccia utente}

La web app deve essere dotata di: 
\begin{itemize}
    \item un sistema di autenticazione (Signup/Login/Logout)
    \item un'interfaccia per la creazione e la gestione dei device
    \item un'interfaccia per la creazione e la gestione dei progetti
    \item un emulatore per testare i progetti in locale
\end{itemize}

\subsubsection{Interfaccia di sviluppo}

Il sistema deve fornire un'API agli sviluppatori per creare i propri applicativi e interagire con i device.
Essa deve:
\begin{itemize}
    \item permettere di ricevere eventi dalla rete
    \item mettere a disposizione i metodi per interagire con hardware e sensori
    \item permettere l'interazione in tempo reale con i device
\end{itemize}

\subsection{Requisiti di performance}

Il sistema deve risultare reattivo e veloce, con bassa latenza e alta affidabilità anche in caso di carichi elevati.
La natura real-time del sistema richiede che le comunicazioni avvengano in tempo reale e che i dati siano aggiornati
in modo continuo.

\subsection{Requisiti non funzionali}

L'interfaccia utente deve essere intuitiva e facile da usare, con un design moderno e accattivante al fine di garantire 
un' ottimale user experience.

L'attendibilità del sistema è fondamentale in quanto deve essere sempre disponibile e funzionante.

Inoltre la supportabilità del sistema è un requisito importante poiché esso deve essere facilmente
manutenibile e aggiornabile per rispondere alle esigenze del mercato.

\subsection{Analisi di mercato}

I segmenti di mercato sono stati analizzati nel capitolo precedente (vedi Capitolo 9)


\section{Conclusioni}

Il sistema si è dimostrato in grado di soddisfare i requisiti richiesti e nei test effettuati si è rivelato
affidabile, performante e scalabile. L'interfaccia utente è stata apprezzata dai tester di tutte le età 
e capacità tecniche per la sua intuitività e facilità d'uso. 

\section{Sviluppi futuri}

Il sistema potrebbe essere migliorato con l'aggiunta di nuove funzionalità quali il supporto per reti di 
comunicazione di tipo mesh oppure quello per protocolli di comunicazione già esistenti come MQTT e CoAP.
Inoltre, potrebbe essere interessante esplorare la possibilità di integrare il sistema con tecnologie emergenti
come la blockchain per offrire un livello aggiuntivo di affidabilità e sicurezza.
